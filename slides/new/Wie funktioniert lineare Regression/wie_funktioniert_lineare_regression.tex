% Options for packages loaded elsewhere
\PassOptionsToPackage{unicode}{hyperref}
\PassOptionsToPackage{hyphens}{url}
%
\documentclass[
  ignorenonframetext,
  t,
  aspectratio=169]{beamer}
\usepackage{pgfpages}
\setbeamertemplate{caption}[numbered]
\setbeamertemplate{caption label separator}{: }
\setbeamercolor{caption name}{fg=normal text.fg}
\beamertemplatenavigationsymbolsempty
% Prevent slide breaks in the middle of a paragraph
\widowpenalties 1 10000
\raggedbottom
\setbeamertemplate{part page}{
  \centering
  \begin{beamercolorbox}[sep=16pt,center]{part title}
    \usebeamerfont{part title}\insertpart\par
  \end{beamercolorbox}
}
\setbeamertemplate{section page}{
  \centering
  \begin{beamercolorbox}[sep=12pt,center]{part title}
    \usebeamerfont{section title}\insertsection\par
  \end{beamercolorbox}
}
\setbeamertemplate{subsection page}{
  \centering
  \begin{beamercolorbox}[sep=8pt,center]{part title}
    \usebeamerfont{subsection title}\insertsubsection\par
  \end{beamercolorbox}
}
\AtBeginPart{
  \frame{\partpage}
}
\AtBeginSection{
  \ifbibliography
  \else
    \frame{\sectionpage}
  \fi
}
\AtBeginSubsection{
  \frame{\subsectionpage}
}
\usepackage{amsmath,amssymb}
\usepackage{lmodern}
\usepackage{iftex}
\ifPDFTeX
  \usepackage[T1]{fontenc}
  \usepackage[utf8]{inputenc}
  \usepackage{textcomp} % provide euro and other symbols
\else % if luatex or xetex
  \usepackage{unicode-math}
  \defaultfontfeatures{Scale=MatchLowercase}
  \defaultfontfeatures[\rmfamily]{Ligatures=TeX,Scale=1}
\fi
\usetheme[]{Rochester}
\usecolortheme{dove}
% Use upquote if available, for straight quotes in verbatim environments
\IfFileExists{upquote.sty}{\usepackage{upquote}}{}
\IfFileExists{microtype.sty}{% use microtype if available
  \usepackage[]{microtype}
  \UseMicrotypeSet[protrusion]{basicmath} % disable protrusion for tt fonts
}{}
\makeatletter
\@ifundefined{KOMAClassName}{% if non-KOMA class
  \IfFileExists{parskip.sty}{%
    \usepackage{parskip}
  }{% else
    \setlength{\parindent}{0pt}
    \setlength{\parskip}{6pt plus 2pt minus 1pt}}
}{% if KOMA class
  \KOMAoptions{parskip=half}}
\makeatother
\usepackage{xcolor}
\IfFileExists{xurl.sty}{\usepackage{xurl}}{} % add URL line breaks if available
\IfFileExists{bookmark.sty}{\usepackage{bookmark}}{\usepackage{hyperref}}
\hypersetup{
  hidelinks,
  pdfcreator={LaTeX via pandoc}}
\urlstyle{same} % disable monospaced font for URLs
\newif\ifbibliography
\usepackage{longtable,booktabs,array}
\usepackage{calc} % for calculating minipage widths
\usepackage{caption}
% Make caption package work with longtable
\makeatletter
\def\fnum@table{\tablename~\thetable}
\makeatother
\setlength{\emergencystretch}{3em} % prevent overfull lines
\providecommand{\tightlist}{%
  \setlength{\itemsep}{0pt}\setlength{\parskip}{0pt}}
\setcounter{secnumdepth}{-\maxdimen} % remove section numbering
\usepackage[ngerman]{babel}
\usepackage{mdframed}
\newmdtheoremenv{ndef}{Definition}
\newmdtheoremenv{nsatz}{Satz}
\def\begincols{\begin{columns}}
\def\begincol{\begin{column}}
\def\endcol{\end{column}}
\def\endcols{\end{columns}}
\usepackage{emoji}
\ifLuaTeX
  \usepackage{selnolig}  % disable illegal ligatures
\fi

\author{}
\date{\vspace{-2.5em}}

\begin{document}

\begin{frame}{Ausgangssituation}
\protect\hypertarget{ausgangssituation}{}
\begin{columns}[T]
\begin{column}{0.48\textwidth}
Wir haben Daten erhoben:

\begin{longtable}[]{@{}lrrrrr@{}}
\toprule
\endhead
x & -0.5 & -1.5 & 0.5 & 1.5 & 2.5 \\
y & -2.0 & -10.0 & 3.5 & 4.0 & 12.0 \\
\bottomrule
\end{longtable}

\pause

Fragen:

\begin{itemize}
  \item Wie hängen \texttt{x} und \texttt{y} zusammen?
  \pause
  \begin{itemize}
    \item[] Korrelation: \texttt{cor(x,y)} = 0.97
    \item[] Positiver Zusammenhang! \emoji{face-with-monocle}
  \end{itemize}
  \pause
  \item Wie können wir einen bestimmten Wert von \texttt{y} vorhersagen?
  \pause
  \begin{itemize}
    \item[] Wir brauchen ein Modell!
  \end{itemize}
\end{itemize}
\end{column}

\begin{column}{0.48\textwidth}
\pause

\includegraphics[height=0.7\textheight]{wie_funktioniert_lineare_regression_files/figure-beamer/unnamed-chunk-3-1}

\centerline{Ein lineares Modell scheint passend...}
\end{column}
\end{columns}
\end{frame}

\begin{frame}{Lineares Modell (\textbf{S}imple \textbf{L}inear
\textbf{R}egression Model)}
\protect\hypertarget{lineares-modell-simple-linear-regression-model}{}
\begin{columns}[T]
\begin{column}{0.48\textwidth}
\[y_i = \beta_0 + \beta_1 x_i + u_i, \quad i = 1,\dots,n\]
\end{column}

\begin{column}{0.48\textwidth}
\vspace{-1cm}
\begin{center}

```
#> Warning: Paket 'latex2exp' wurde unter R Version 4.2.2 erstellt
```

```
#> Warning in is.na(x): is.na() auf Nicht-(Liste oder Vektor) des Typs 'expression'
#> angewendet
```


\includegraphics[height=0.55\textheight]{wie_funktioniert_lineare_regression_files/figure-beamer/unnamed-chunk-4-1} 
\end{center}
\end{column}

\pause
\end{columns}

\vspace{-1.7cm}
\small
\begin{itemize}
  \item $i$ ist der Index für eine Beobachtung
  \item $n$ ist die Anzahl Beobachtungen
  \item $y_i$ ist die abhängige (zu erklärende) Variable für die Beobachtung $i$
  \item $x_i$ ist der Regressor (die erklärende Variable) für die Beobachtung $i$
  \item $u_i$ ist der Fehlerterm (der Messfehler) für die Beobachtung $i$
  \pause
  \[
  \left.\parbox{0.57\textwidth}{
      \item $\beta_0$ ist der Achsenabschnitt (auch Intercept genannt) 
      \item $\beta_1$ ist der Steigungsparameter
  }\right\} \text{ Parameter, die wir schätzen wollen}
  \]
\end{itemize}
\normalsize
\end{frame}

\begin{frame}{Modellannahmen}
\protect\hypertarget{modellannahmen}{}
\vspace{-0.5cm}
\small
\begin{center}
\renewcommand{\arraystretch}{1.5}
\begin{tabular}{lp{2cm}p{4cm}p{5cm}}
       & Stichwort                         & Annahme & wäre z.B.\ verletzt, wenn... \\ \hline
 SLR.1 & \textbf{Modell}                   & $y = \beta_0 + \beta_1 x + u$ mit den Parametern $\beta_0,\ \beta_1 \in \mathbb{R}$ & der Zusammenhang ist exponentiell ($y = \beta_0 e^{\beta_1 x} + u$)\\
 SLR.2 & \textbf{Stichprobe}               & $\{(y_i , x_i ), i = 1, \dots, n\}$ zufällig gemäß SLR.1 generiert & aus der Grundgesamtheit der Wahlberechtigten wurden nur Studierende befragt\\
 SLR.3 & \textbf{Information im Regressor} & $\text{Var}(x) > 0$ & Experiment immer mit den exakt gleichen Parametern durchgeführt \\
 SLR.4 & \textbf{Bedingte Erwartung}       & $\mathbb{E}(u\mid x) = 0$ & es gibt einen systematischen Messfehler \\
 SLR.5 & \textbf{Homoskedastizität}        & $\text{Var}(u\mid x)=\sigma^2$ & die Körpergröße von Kleinkindern hat eine geringere Varianz als die von Erwachsenen\\
\end{tabular}
\end{center}
\normalsize
\end{frame}

\begin{frame}{Schätzwerte bestimmen}
\protect\hypertarget{schuxe4tzwerte-bestimmen}{}
\begin{columns}[T]
\begin{column}{0.48\textwidth}
``Kleinste Quadrate'' Methode: \begin{align*}
  {\arg \min}_{\hat\beta_0, \hat\beta_1} \sum_{i = 1}^n (\underbrace{y_i - \underbrace{\hat\beta_0 + \hat\beta_1 x_i}_{\text{Vorhersage } \hat y_i}}_{\text{Residuum } \hat u_i})^2
\end{align*}

Schätzer: \begin{align*}
  \hat\beta_0 &= \bar{Y} - \hat\beta_1 \bar{X} \\
  \hat\beta_1 &= \frac{\text{Cov}(Y,X)}{\text{Var}(X)}
\end{align*}
\end{column}

\begin{column}{0.48\textwidth}
\includegraphics[height=0.7\textheight]{wie_funktioniert_lineare_regression_files/figure-beamer/unnamed-chunk-5-1}
\end{column}
\end{columns}
\end{frame}

\begin{frame}{Mehrere Regressoren (\textbf{M}ultiple \textbf{L}inear
\textbf{R}egression Model)}
\protect\hypertarget{mehrere-regressoren-multiple-linear-regression-model}{}
Statt einem können wir uns auch \(K\) Regressoren gleichzeitig
anschauen:

\[y_i = \beta_0 + \beta_1 x_{i,1} + \ldots + \beta_K x_{i,K} + u_i, \quad i = 1,\dots,n\]

Wie werden die Parameter interpretiert?

\begin{itemize}
  \item Achsenabschnitt $\beta_0$: 
  \begin{itemize}
    \item[] Welches $y$ erwarten wir, wenn alle Regressoren Null sind?
  \end{itemize}
  \item Steigungsparameter $\beta_k$, $k = 1,\dots,K$:
  \begin{itemize}
    \item[] Wie verändert sich $y$, wenn sich $x_k$ um eine Einheit verändert und alle anderen Regressoren konstant gehalten werden? (ceteris paribus Analyse)
  \end{itemize}
\end{itemize}
\end{frame}

\begin{frame}{Welche Annahmen treffen wir?}
\protect\hypertarget{welche-annahmen-treffen-wir}{}
\vspace{-0.5cm}
\small
\begin{center}
\renewcommand{\arraystretch}{1.5}
\begin{tabular}{lp{4cm}p{7cm}}
       & Stichwort                         & Annahme  \\ \hline
 MLR.1 & \textbf{Modell}                   & $y = \beta_0 + \beta_1 x_1 + \dots + \beta_K x_K + u$ mit den Parametern $\beta_0,\ \beta_1, \dots, \beta_K\ \in \mathbb{R}$ \\
 MLR.2 & \textbf{Stichprobe}               & $\{(y_i , x_{1,i}, \dots, x_{K,i} ), i = 1, \dots, n\}$ zufällig gemäß MLR.1 generiert \\
 MLR.3 & \textbf{Information im Regressor} & $\text{Var}(x_k) > 0$ für alle $k=1,\dots,K$  \\
 MLR.4 & \textbf{Bedingte Erwartung}       & $\mathbb{E}(u\mid x_1,\dots,x_K) = 0$  \\
 MLR.5 & \textbf{Homoskedastizität}        & $\text{Var}(u\mid x_1,\dots,x_K)=\sigma^2$ \\
\end{tabular}
\end{center}
\normalsize
\end{frame}

\begin{frame}{Wie werden die Parameter geschätzt?}
\protect\hypertarget{wie-werden-die-parameter-geschuxe4tzt}{}
\begin{itemize}
\tightlist
\item
  Was ist die Normalgleichung? Und was bedeutet Multikollinearität?
\end{itemize}
\end{frame}

\begin{frame}{Unser Programm:}
\protect\hypertarget{unser-programm}{}
Heute:

\begin{itemize}
\tightlist
\item
  Wie führt man eine (multiple) lineare Regression in \texttt{R} durch?
\item
  Wie interpretiert man den \texttt{R} Output einer (multiplen) linearen
  Regression?
\end{itemize}

Nächste Woche:

\begin{itemize}
\tightlist
\item
  Wovon hängt die Varianz der Schätzer im MLR Modell ab?
\item
  Was besagt das Gauss-Markov-Theorem?
\item
  Was ist der Unterschied zwischen Korrelation und Kausalität?
\end{itemize}
\end{frame}

\end{document}
